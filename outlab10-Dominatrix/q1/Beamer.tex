\documentclass{beamer}
\usepackage{graphicx}
\usepackage{amsmath}
\usepackage{kpfonts}
\usepackage{boxedminipage}
\usepackage{bcprules}
\usepackage{tikz}
\usetheme{CambridgeUS}
\usetikzlibrary{positioning,arrows}
\usetikzlibrary{shapes.geometric}
%%%%%%%%%%%%%%%%Title Page%%%%%%%%%%%%%%%%%%%%%%%%%%%%%%%%%

\title[Lambda Calculus]{Introduction to Beamer}
\subtitle[]{}
\author[F. last]{Akkapaka Saikiran}
\institute[IITB]{
  Department of Computer Science and Engineering\\
  IIT Bombay.\\
  Powai, Mumbai - 400076\\[1ex]
  \texttt{psycherun@cse.iitb.ac.in}
}
\date[\today]{\today}

%%%%%%%%%%%%%%%%%%%%%%%%%%%%%%%%%%%%%%%%%%%%%%%%%%%%%%%%%%%


\newtheorem{exercise}{Exercise}


\begin{document}
%--- the titlepage frame -------------------------%
\begin{frame}[plain]
  \titlepage
\end{frame}
\tikzstyle{rectanglebox} = [draw, rectangle, minimum height=6em, minimum width=10.5em]
    \tikzstyle{new} = [coordinate]
%%%%%%%%%%%%%%%%%%%%%%%%%%%%%%%%%%%%%%%%%%%%%%%%%%%%%%%%%%%
\begin{frame}[fragile]{\bf  This is the title}
Beamer is a \LaTeX \:class for preparing presentations.

\begin{enumerate}
\item Slides are called frames in Beamer.
\item This is the usual ordered list in \LaTeX.
\item Following slides will contain random content which will show you various ways of using it. You need to replicate it.
\item Of course! we will give you boilerplate code!
\end{enumerate}
\end{frame}

\begin{frame}[fragile]{Type Rules}
\begin{figure}[htb]
\includegraphics[width=0.75\textwidth]{type-lattice.pdf}
\caption{This is the caption}
\end{figure}

\end{frame}

\begin{frame}[fragile]{Type Rules}

\begin{itemize}
    \item A substitution is a list of pairs denoted as $ S = \{\alpha_1 /\tau_1 \ldots \alpha_n/\tau_n \}$ .
    \item A substitution $S$ applied on a type expression $\sigma$, denoted by $S (\sigma )$ involves simultaneous substitution of the variables $ \alpha_1 \ldots \alpha_n$, if they occur free in $\sigma$, by the corresponding type expressions $\tau_1\ldots \tau_n$.

    \begin{block}{Definition}
    Let $\sigma = \forall \alpha_1 \ldots \alpha_m .\tau$ and $\sigma_0 = \forall \beta_1 \ldots \beta_n.\tau_0 $. Then $\sigma_0$ is a \textit{generic instance} of $\sigma$ , iff there is a substitution $S$ acting only on $\{\alpha_1 \ldots \alpha_m \}$ such that $\tau^{\prime}$ = $S(\tau)$ and no $\beta_i$ is free in $\sigma$.
    \end{block}
   
    \item Clearly, the restriction that no $\beta_i$ is free in $\sigma$ is needed, else we would have absurdities like $\alpha \rightarrow Int \le \forall \alpha.\alpha \rightarrow Int $.
   
    \end{itemize}

\end{frame}


\begin{frame}[fragile]{\bf Recapitulation – Type rules for $\lambda_2$}

\begin{equation}
    \Gamma \cup \{x :: \sigma \} \vdash x :: \sigma \tag{\sc Var}
\end{equation}  
\vspace{-5pt}
\begin{equation}
    \Gamma \cup \{x :: \sigma \} \vdash c :: \sigma \tag{\sc Con}
\end{equation}
\vspace{0pt}
\begin{equation}
    \frac{\Gamma \vdash M :: \sigma \; \; \; \; \; \; \; \; \sigma^\prime \ge \sigma}
    {\Gamma \vdash M :: \sigma^\prime} \tag{\sc Inst}
\end{equation}
\vspace{5pt}
\begin{equation}
    \frac{\Gamma \vdash M :: \sigma \; \; \; \; \; \; \; \; \alpha \notin FV(\Gamma)}
    {\Gamma \vdash M :: \forall \alpha.\sigma} \tag{\sc Gen}
\end{equation}
\vspace{5pt}
\begin{equation}
    \frac{\Gamma \vdash M :: \tau_1 \rightarrow \tau_2 \; \; \; \; \; \; \; \; \Gamma \vdash N :: \tau}
    {\Gamma \vdash M \; \:  N :: \tau_2} \tag{\sc M-App}
\end{equation}
\vspace{5pt}
\begin{equation}
    \frac{\Gamma,x :: \tau_1 \vdash M :: \tau_2}
    {\Gamma \vdash \lambda x.M :: \tau_1 \rightarrow \tau_2} \tag{\sc M-Abs}
\end{equation}
   
\end{frame}


\begin{frame}[fragile]{\bf Hindley-Milner - Type checking applications}
\setbeamertemplate{enumerate items}[default]
\begin{enumerate}[1:]
    \item t \textbf{is a variable} x
    \begin{tikzpicture}
    
    \node [new]                 (input_lines)     {};
    \node [rectanglebox, right=6em of input_lines ]   (model)     {\( x \)};
    \node [new, right=6em of model, yshift=2em] (output1)    {};
    \node [new, right=6em of model, yshift=-2em] (output2)  {};
    \node [below=0.2em of model](figure caption){$\Gamma x = \forall\alpha_{1},\ldots\alpha_{n}\cdot\tau$};
    \draw (input_lines) edge[->,thick,>=stealth] node[above, xshift=-2.3em]{$\Gamma$} (model);
    \draw (model)edge[->,thick,>=stealth, out=20,looseness=0]node[above, xshift=4em]{ $\tau[\alpha_{1}/\beta_{1},\ldots \alpha_{n}/\beta_{n}], ~\beta_{1}, \ldots \beta_{n} \notin \tau$}(output1);
    \draw (model)edge[->,thick,>=stealth, out=-20,looseness=0]node[below, xshift=4em] {{The identity substitution $\theta_{id}$}} (output2);
\end{tikzpicture}
   
    \begin{itemize}
        \item $\beta_1 , \ldots , \beta_n$ are fresh variables.
        \item Reason for monomorphising the type of $x$: We try to find the type of a variable only in the context of an application, and our application is monomorphic.
    \end{itemize}
\end{enumerate}
       
\end{frame}

\begin{frame}[fragile]{\bf Hindley-Milner - Type checking applications}
\setbeamertemplate{enumerate items}[default]
\begin{enumerate}[1]
    \item Typecheck $e_1$ with the initial environment $\Gamma$. Result is $\tau_1$ and $\theta_1$.
    \item Typecheck $e_2$ with the environment $\theta_1\Gamma$. Result is $\tau_2$ and $\theta_2$.
    \item Unify $\theta_2 \tau_1$ and $\tau_2 \rightarrow \alpha$. Assume that unifier is $\theta$. And the unified term $(\theta \; \alpha)$ is $\tau_3$.
    \item Type of the application is $\tau_3$ and the final substitution is $\theta \circ \theta_2 \circ \theta_1$.
\end{enumerate}

\end{frame}

\end{document}


